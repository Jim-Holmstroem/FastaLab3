\documentclass[a4paper,twoside=false,abstract=false,numbers=noenddot,
titlepage=false,headings=small,parskip=half,version=last]{scrartcl}
\usepackage[utf8]{inputenc}
\usepackage[T1]{fontenc}
\usepackage[english]{babel}
\usepackage[colorlinks=true, pdfstartview=FitV,
linkcolor=black, citecolor=black, urlcolor=blue]{hyperref}
\usepackage{verbatim}
\usepackage{graphicx}
\usepackage{multirow}

\usepackage{tikz}
\usetikzlibrary{matrix}

\usepackage{amsmath}
\usepackage{amsthm}
\usepackage{amssymb}
\usepackage{amsfonts}

\usepackage{float}

\usepackage{gensymb}

\usepackage{authblk}

\usepackage{helpers}


\title{Solid State Physics - IM2601}
\subtitle{Laboration 3}
    \author[1]{Fredrik Forsberg}
    \author[1]{Jim Holmström}
    \author[1]{Samuel Zackrisson}
    \affil[1]{Engineering Physics, Royal Institute of Technology}
    \affil[1]{\{fforsber, jimho, samuelz\}@kth.se}


\begin{document}
\maketitle
\thispagestyle{empty}


In order to harness energy from light, a variant of the diode can be used. The
photovoltaic cell, or solar cell, is a large version of the p-n junction, where a thin net of conducting material
which don't obscure the sunlight is placed on top of the p- and n-doped semiconductors,
who in turn are connected to a metal surface. This result in a large p-n junction.
On the n-side of the junction there are atom which can donate electrons while there are acceptor
atoms on the p-side.
This results in a permanent electric field over the junction which can stop the flow of electrons if the voltage is too low.

When the energy of the light hitting the solar cell is greater than the band gap of the diode, $E_g$, electrons will be excited and a current will start to flow in the direction opposite the electric field.

In this laboratory experiment we examine a thin film of polycristalline silicon (Si). At a temperature of $T = 300 K$ the band gap of a silicon diode is $E_g=1.11 eV$. The minimum frequency of light which will be able to create a current is given by

\begin{equation}
    \nu = \frac{E_g}{h} \approx 2.689*10^{14} Hz
\end{equation}

where $h$ is the Planck constant.

\section{Discussion of the results}

We use a Power Cassy$\textsuperscript{TM}$ in order to measure the current under applied bias and during illumination by a standard desk lamp.
When no bias is applied

\plotpdf{distancedepence}{Generated current over distance from light source with
zero bias (blue) and generated current for only ambient light (red).}


\section{Fitting experimental unilluminated diode data to the diode equation}
The data collected can be used to verify the diode equation,

\begin{equation}
    I = I_0 \left( 1 - e^{-V/V_0}\right)\label{eq:diode-equation}
\end{equation}

where $V_0 = \frac{nk_BT}{e}$.
$T$ is the diode temperature, $e$ the elementary charge and $k_B$ the Boltzmann constant.
$n$ is the \emph{ideality factor} which usually lies in the range $[1,2]$ or sometimes higher for low currents.
It accounts for imperfections in the diode under study, recombinations of electrons and holes.
The data collected for an unilluminated solar cell was fit to equation \eqref{eq:diode-equation} by minimizing the sum of squared errors.
The resulting coefficients were $I_0=3.34\,\mu A$, $V_0=53.3\, mV$.
This value for $V_0$ yields an ideality factor at room temperature $T=293\,K$ of $n = 2.11$.
This is slightly above the expected interval, possibly due to the low current.
The manner of fitting the data could also be discussed, as least-square fitting an exponential will mainly be fit to the data points with the largest (in magnitude) exponent.
This can be observed in figure \ref{fig:diode-equation-fit}, where the experimental current is distinctly larger than the fitted current.
The mean current for voltages $V>-0.2$ is $1.54\,mA$.
\plotpng{diode-equation-fit}{Plot of the experimental IV-data for the unilluminated diode, together with the graph of the fitted diode equation.}
\section{Discussion of the results}


\section{Efficiency}
A simple model is set up in \cite{lab-instruction} for a theoretical upper limit to the efficiency $\nu$ of the solar cell.
he $pn$-junction is assumed to be operating at $0\,K$, with a band gap $E_g$ and illuminated by a blackbody at temperature $T$.
The efficiency is a function of $x_g=E_g/(k_BT)$, and a plot in \cite{lab-instruction} is reproduced here as figure \ref{fig:efficiency-plot}.
\plotpng{efficiency-plot}{Theoretical upper limit of the efficiency $\nu$ of a $0\,K$ $pn$-junction solar cell with band gap $E_g$ and illuminated by a temperature $T$ blackbody.}

The maximal theoretical efficiency appears to be about $\eta_{max}=0.44$, at $x_g=2.26$.
Treating the sun as a $5800\,K$ blackbody gives an optimal band gap of $E_g =
x_g k_B T = 1.13\,eV$.
An appropriate crystal near this $E_g$ can be found using Table 1 from chapter 8 of \cite{Kittel}.
A $1.13\,eV$ band gap at $0\,K$ corresponds best to a Si crystal, with band gap
$1.17\,eV$ at $0\,K$.
Assuming this had still been optimal at $300\,K$, Si remains closest with a
$1.11\,eV$ band gap.\\
In this laboratory experiment there is no sun. The desk lamp is instead assumed to radiate like a blackbody at temperature $3000\,K$. The Si solar cell will then here have $x_g = 4.53$, corresponding to a theoretical maximum possible efficiency of $\approx 0.25$.\\
The actual efficiency of the solar cell in this lab can be roughly estimated.
Assuming the light bulb is a $P_0=60\,W$ light bulb emitting all of its power
like a $3000\,K$ blackbody, at a distance $R$ the solar cell with an area
$A=6\,cm^2$ will be receiving an input power $P_{in} = P_0 \frac{A}{2\pi R^2}
= \frac{57.2\,Wcm^2}{R^2}$.
The maximal observed power output $P_{out}$ for different distances $R$ give a range of efficiencies plotted in figure \ref{fig:real-efficiency-plot}.
All of these measurements are overestimates.
For large distances the background light probably contributes considerably to the input power, and for smaller distances the half-sphere approximation is not realistic.
The lamp screen scatters the light in a narrower cone shape and thus the input power larger.\\
Assuming the smallest distances are the most controlled and correct, the measured efficiency lands at $\approx 20\,\%$. This is below the theoretical maximum, but not by much. There are many uncontrolled contributions to the efficiency calculation.
\plotpng{real-efficiency-plot}{Estimated maximal observed efficiencies of the solar cell for different distances to the desk lamp.}

\section{Relation between maximum power and the distance between lamp and solar cell}
The maximum power sent out by the lamp, $P_{max}$, in relation to the distance from said lamp, $d$, can be modeled as being spread out equally over the area of a sphere with the radius $d$. The area of the solar cell was measured to be $a=0.001475 m^{2}$. $\delta$ represents a constant systematic error of the distance measurement and $k’$ is a constant proportional to the electrical power of the lamp and how good the reflectors of the lamp are.

\begin{equation}
    P_{max}(d) = k’\frac{a}{4\pi(d+\delta)^2} = \frac{k}{(d+\delta)^2}
\end{equation}

The two unknown constants $k$ and $\delta$ can be calculated performing a nonlinear fit of the maximum power at 10 different distances in MATLAB. This results in $k = 0.0004 Wm^{2}$ and $\delta = -0.0319 m$. The distances of measurement were chosen such that the points were placed equidistant apart in order to ensure linear relationships.

\plotpng{PmaxVSd}{Plot of the experimental data and the corresponding linear equation in relation to the inverted distance squared.}

Below are the resulting relation between the voltage and the average current respectively the average power plotted when the applied bias is a triangular wave with amplitude $0.57 V$ and a frequency of $0.2 Hz$.

\plotpng{IVsU}{Plot of the average current  at different distances from the lamp in relation to the applied voltage.}

\plotpng{PVsU}{Plot of the average power at different distances from the lamp in relation to the applied voltage.}


\section{Optimal load of the solar cell}
The solar cell connected to the Power Cassy$\textsuperscript{TM}$ can be roughly modeled as shown in the figure below.

\plotpng{CircuitModel}{A very simplified model of the Power Cassy$\textsuperscript{TM}$ connected to the solar cell.}

Given the directions shown in the figure we have $U=U_0-IR_0$ such that the applied voltage has the same effect as a resistor $R=-frac{U}{I}$. The power done on the resistor is then $P=-U I=-R I^{2}$. This results in that we can write the current in terms of the gained voltage $U_0$ and the resistances as $I=\frac{U_0}{R + R_0}$. The resulting power given the direction of the current is thus

\begin{equation}
    P = R (\frac{ U_0}{ R_0+R})^{2}
\end{equation}

In order to find the resistance of the load which results in the maximum power, we take the derivative $\frac{d P}{d R} = 0$. Given the direction of the current, the optimal resistance is $R = R_0$. The corresponding optimal voltage is

\begin{equation}
    U_{opt}=R_0 I=\frac{U_0}{2}
\end{equation}

The maximum power output at the resistor results in the voltage $ U_{opt}$ which corresponds to half of the voltage over the solar cell. Theoretically this means that $U_{opt} = 0.285 V$. When $I=0$ in the I/U-plot above, we must have that $U = U_0$. This can be seen to be approximately the case at close distances.

\begin{thebibliography}{1}
    \bibitem{Kittel}
        Charles Kittel,
        {\em Introduction to Solid State Physics 8th Edition},
        2005.
	\bibitem{lab-instruction}
		Rickard Fors,
		{\em Photovoltaic effect: Diode IV-characteristics. Laboration Condensed Matter Physics / Solid State Physics},
		2005.
\end{thebibliography}


\end{document}
